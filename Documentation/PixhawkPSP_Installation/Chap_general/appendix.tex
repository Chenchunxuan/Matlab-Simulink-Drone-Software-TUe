%!TEX root = report.tex

%% Add normal figure:
%%%%%%%%%%%%%%%%%%%%%
%\begin{figure} [H]
%  \begin{center}
%    \includegraphics[width=0.70\linewidth]{Figures/Ex1_excitation_ut.pdf}
%    \caption{The caption}
%    \label{fig:Ex1_excitation_ut}
%  \end{center}
%\end{figure}
%%%%%%%%%%%%%%%%%%%%%

%% Add subfigure:
%%%%%%%%%%%%%%%%%%%%%
%\begin{figure}[H]
%\begin{center}
%	\begin{subfigure}{0.49\textwidth} 	
%    \includegraphics[scale=0.32]{Figures/Ass2_Bode_like_Sens.pdf}
%    \caption{Bode-like plot of the Sensitivity}
%    \label{fig:Ass2_Bode_like_Sens}
%	\end{subfigure} 	
%	\begin{subfigure}{0.49\textwidth}
%    \includegraphics[scale=0.32]{Figures/Ass2_Bode_like_Comp_Sens.pdf}
%    \caption{Bode-like plot of the Complementary Sensitivity}
%    \label{fig:Ass2_Bode_like_Comp_Sens}
%	\end{subfigure}
%\end{center}
%	\caption{}
%	\label{fig: Ass2_Bode_like_Comp_Sens_overview}
%\end{figure}
%%%%%%%%%%%%%%%%%%%%%
\chapter{Auto and cross-correlation functions.}
\label{App:Correlation_Tests}
\begin{figure}[H]
\begin{center}
	\begin{subfigure}{0.49\textwidth} 	
    \includegraphics[scale=0.49]{Figures/System_Identification/Correlation/Cor1}
    \caption{Auto-correlation $y_1$ and $y_2$.}
	\end{subfigure} 	
	\begin{subfigure}{0.49\textwidth}
    \includegraphics[scale=0.49]{Figures/System_Identification/Correlation/Cor2}
    \caption{Auto-correlation $y_3$ and cross-correlation $y_1$ $u_1$.}
	\end{subfigure}
\end{center}
\begin{center}
	\begin{subfigure}{0.49\textwidth} 	
    \includegraphics[scale=0.49]{Figures/System_Identification/Correlation/Cor3}
    \caption{Cross-correlation $y_1$ $u_2$ and $y_1$ $u_3$.}
	\end{subfigure} 	
	\begin{subfigure}{0.49\textwidth}
    \includegraphics[scale=0.49]{Figures/System_Identification/Correlation/Cor4}
    \caption{Cross-correlation $y_1$ $u_4$ and $y_1$ $u_5$.}
	\end{subfigure}
\end{center}
\end{figure}
\begin{figure}[H]
\begin{center}
	\begin{subfigure}{0.49\textwidth} 	
    \includegraphics[scale=0.49]{Figures/System_Identification/Correlation/Cor5}
	\caption{Cross-correlation $y_2$ $u_1$ and $y_2$ $u_2$.}
	\end{subfigure} 	
	\begin{subfigure}{0.49\textwidth}
    \includegraphics[scale=0.49]{Figures/System_Identification/Correlation/Cor6}
    \caption{Cross-correlation $y_2$ $u_3$ and $y_2$ $u_4$.}
	\end{subfigure}
\end{center}
\begin{center}
	\begin{subfigure}{0.49\textwidth} 	
    \includegraphics[scale=0.49]{Figures/System_Identification/Correlation/Cor7}
    \caption{Cross-correlation $y_2$ $u_5$ and $y_3$ $u_1$.}
	\end{subfigure} 	
	\begin{subfigure}{0.49\textwidth}
    \includegraphics[scale=0.49]{Figures/System_Identification/Correlation/Cor8}
    \caption{Cross-correlation $y_3$ $u_2$ and $y_3$ $u_3$.}
	\end{subfigure}
\end{center}
\begin{center}
	\begin{subfigure}{0.49\textwidth} 	
    \includegraphics[scale=0.49]{Figures/System_Identification/Correlation/Cor9}
    \caption{Cross-correlation $y_3$ $u_4$ and $y_3$ $u_5$.}
	\end{subfigure} 	
\end{center}
\end{figure}


\chapter{Finite state machine} \label{App:FSM}
The finite state machine used when $h_2<h_3$ and $h_1>h_3$ can be seen in Figure \ref{fig:FSM2}.

\begin{figure} [H]
  \begin{center}
    \includegraphics[width=0.60\linewidth]{Figures/Switched_Controller/FiniteStateMachine2.jpg}
    \caption{Finite state machine for valve control when $h_1\geq h_3$}
    \label{fig:FSM2}
  \end{center}
\end{figure}


\chapter{LQR controller} \label{App:LQR}
The simulink model made for this control system can be seen in Figure \ref{fig:LQRsim}.

\begin{figure} [H]
  \begin{center}
    \includegraphics[width=1\linewidth]{Figures/Switched_Controller/LQRsimulink.PNG}
    \caption{Simulink model for the LQR controller}
    \label{fig:LQRsim}
  \end{center}
\end{figure}

The Matlab code in the 'select LQR' block can be seen below. This function block determines the current state of the system using the current valve positions of valve one and two. Then the correct control matrix will be chosen and multiplied with the state-reference to get the desired input of the system.

\begin{lstlisting}
function [Q1,Q2,oldrefout] = fcn(h1,h2,h3,h10,h20,h30,v1,v2,oldrefin)
%#codegen
    %Diameter = 0.14; %in meters
    %A = ((Diameter^2)*pi)/4; %in meters squared
K11 = [0.1536   -0.0000    0.0345
   -0.0000    0.1536   -0.0345];
K12 = [0.1536   -0.0000    0.0000
   -0.0000    0.1647   -0.0682];
K21 = [0.1495   -0.0000    0.0739
   -0.0000    0.1536   -0.0000];
K22 = [0.1480   -0.0021    0.0431
   -0.0019    0.1633   -0.0403];

xk = [h1-h10; h2-h20; h3-h30];


if h10 == oldrefin(1) && h20 == oldrefin(2) && h30 == oldrefin(3) && oldrefin(4) == 0;
    %% Select K based on different valve positions (v1,v2 ==> 1e-8 and 0.2)
    if v1 < 0.02 && v2 < 0.02
    input = -K11*xk;
    else if v1 < 0.02 && v2 >= 0.02
            input = -K12*xk;
        else if v1 >= 0.02 && v2 < 0.02
                input = -K21*xk;
            else
                input = -K22*xk;
            end
        end
    end

    Q1 = input(1,1);
    Q2 = input(2,1);
    newref = 0;
else %reference changed
    if h10 < h1 || h30 < h3
        Q1 = 0;
        Q2 = 0;
        if h10 < h1
            newref = 1;
        else
            newref = 2;
        end
    else
        newref = 0;
        if v1 < 0.02 && v2 < 0.02
            input = -K11*xk;
        else if v1 < 0.02 && v2 >= 0.02
            input = -K12*xk;
            else if v1 >= 0.02 && v2 < 0.02
                    input = -K21*xk;
                else
                    input = -K22*xk;
                end
            end
        end
        Q1 = input(1,1);
        Q2 = input(2,1);
    end
    
end
oldrefout = [h10; h20; h30; newref];
\end{lstlisting}

the code in the 'Determine valve positions' block is shown below. This block implements the finite state machine described. It determines the wanted valve positions and these will be used to calculate the inputs of the valves (Next function block).

\begin{lstlisting}
function [V1,V2,V3] = valvepositions(h1,h2,h3,h1ref,h2ref,h3ref,oldrefin)

V1 = 1;
V11 = 1;
V12 = 1;
V2 = 1;
V21 = 1;
V22 = 1;
V3 = 0;

if h2ref > h3ref
if h3 > 0.5*h3ref
    V11 = 0.8;
    V2 = 0.8;
    if h3 > 0.7*h3ref
        V11 = 0.6;
        V2 = 0.6;
        if h3 > 0.9*h3ref
            V11 = 0.4;
            V2 = 0.4;
            if h3 > 0.95*h3ref
                if h3 > h3ref
                    V11 = 0.1;
                    V2 = 0.1;
                else
                    V11 = 0.4;
                end
            end
        end
    end
end

if h1 > 0.5*h1ref
    V12 = 0.8;
    if h1 > 0.7*h1ref
        V12 = 0.6;
        if h1 > 0.9*h1ref
            V12 = 0.4;
            if h1 > 0.95*h1ref
                if h1 > h1ref
                    V12 = 0.1;
                else
                    V12 = 0.4;
                end
            end
        end
    end
end

    if h2 > h2ref+0.0001
        V3 = 0.4;
    else
        V3 = 0.1;
    end

    V1 = min(V11,V12);
else if h1ref > h3ref        
        
    if h3 > 0.5*h3ref
        V1 = 0.8;
        V21 = 0.8;
        if h3 > 0.7*h3ref
            V1 = 0.6;
            V21 = 0.6;
            if h3 > 0.9*h3ref
                V1 = 0.4;
                V21 = 0.4;
                if h3 > 0.95*h3ref
                    if h3 > h3ref
                        V1 = 0.1;
                        V21 = 0.1;
                    else
                        V1 = 0.4;
                        V21 = 0.4;
                    end
                end
            end
        end
    end

    if h2 > 0.5*h2ref
        V22 = 0.8;
        if h2 > 0.7*h2ref
            V22 = 0.6;
            if h2 > 0.9*h2ref
                V22 = 0.4;
                if h2 > 0.95*h2ref
                    if h2 > h2ref
                        V22 = 0.1;
                    else
                        V22 = 0.4;
                    end
                end
            end
        end
    end


    if h2 > h2ref
        V3 = 0.4;
    else
        V3 = 0.1;
    end

    V2 = min(V21,V22);
    end
end

if oldrefin(4) == 1
    V1 = 1;
    V2 = 1;
    V3 = 1;
end
if oldrefin(4) == 2
    V1 = 0.1;
    V2 = 1;
    V3 = 1;
end

end
\end{lstlisting}

The last code used is to determine the input the the valves. Only valve one two and three will be controlled by the finite state machine so only inputs will be given to open/close these valves which results in 6 outputs of this block. First it checks whether the current valve position if higher/lower then the wanted valve position, if this is true, then it gives an input of 1 to close/open the valve. The last part checks whether the valves are already completely open or closed and when this is the case the valves do not need to be more opened or closed.

\begin{lstlisting}
function [V1Op,V1Cl,V2Op,V2Cl,V3Op,V3Cl] = fcn(V1,V2,V3,V1pos,V2pos,V3pos,Vstate1,Vstate2,Vstate3)
V1Op = 0;
V1Cl = 0;
V2Op = 0;
V2Cl = 0;
V3Op = 0;
V3Cl = 0;

%% Give input signal for opening or closing valve
if V1pos > V1+0.01
    V1Cl = 1;
else if V1pos < V1-0.01
        V1Op = 1;
    end
end

if V2pos > V2+0.01
    V2Cl = 1;
else if V2pos < V2-0.01
        V2Op = 1;
    end
end

if V3pos > V3+0.01
    V3Cl = 1;
else if V3pos < V3-0.01
        V3Op = 1;
    end
end
%% Check whether the vales are already fully open or closed
if V1Cl == 1 && Vstate1 == 0
    V1Cl = 0;
end
if V1Op == 1 && Vstate1 == 1
    V1Op = 0;
end
if V2Cl == 1 && Vstate2 == 0
    V2Cl = 0;
end
if V2Op == 1 && Vstate2 == 1
    V2Op = 0;
end
if V3Cl == 1 && Vstate3 == 0
    V3Cl = 0;
end
if V3Op == 1 && Vstate3 == 1
    V3Op = 0;
end
\end{lstlisting}





\chapter{Individual contribution}\label{App:Ind_contr}
Here the individual contribution from each person participating in the project can be seen.
\section{Parth Sharma}
\begin{itemize}
\item Literature Research: 3 Tank Modelling, Controller designs
\item Controller Tests
\item LQR design with 2 inputs(Unsuccessful)
\item Report Work
\end{itemize}

\section{Tom Albers}
\begin{itemize}
\item Literature study, 3-tank system modelling and system identification.
\item Build mathematical model.
\item System identification.
\item Designing switching controller.
\item Switching controller tests.
\item Results switching controller.

\end{itemize}

\section{Mark in 't Groen}
\begin{itemize}
\item Identification: First principle modeling
\item Controller design: LQR controller
\item Controller design: MPC controller
\item Results: LQR controller
\item Results: MPC controller
\item Performance analysis
\item Recommendation: Change mathematical model, finite state machine, LQR controller and MPC controller
\end{itemize}
\chapter{LQR switching controller stability proof}
The stability of the system under arbitrary switching is guaranteed, if there exist a common Lyapunov function for which the P matrix is positive definite, and the derivative is negative definite.\\
Since the controller inside the switching planes is already calculated by solving the Algebraic Ricatti Equation, see section \ref{sec:LQR}, stability has only to be proven under arbitrary switching.
\begin{eqnarray}
\begin{aligned}
\dot{x}&=Ax+Bu\\
u&=-Kx
\end{aligned}
\end{eqnarray}
A system is stable, if the following conditions hold:
\begin{eqnarray}
\begin{aligned}
V\left(x\right)&=x^TPx>0\\
\dot{V}\left(x\right)&=\dot{x}^TPx+x^TP\dot{x}= x^T\left(\left(A-BK\right)^TP+P\left(A-BK\right)\right)x<0\\
V\left(0\right)&=0\\
V\left(\infty\right)&=\infty
\end{aligned}
\end{eqnarray}
Guaranteeing stability over switching can be proven via multiple methods, for instance finding a common Lyapunov function, or a piecewise quadratic lyapunov function.
Since the stability can be guaranteed by finding a common Lyapunov function, this is the method used.
\begin{eqnarray}
\begin{aligned}
P>0\\
\left(A_\sigma-B_\sigma K_\sigma\right)^TP+P\left(A_\sigma-B_\sigma K_\sigma\right)<0\\
\end{aligned}
\end{eqnarray} 
This common Lyapunov is found with an LMI solver in Matlab, with the following LMI's:
\begin{eqnarray}
\begin{aligned}
\left(A_1-B_1 K_1\right)^TP+P\left(A_1-B_1 K_1\right)<0\\
\left(A_2-B_2 K_2\right)^TP+P\left(A_2-B_2 K_2\right)<0\\
\left(A_3-B_3 K_3\right)^TP+P\left(A_3-B_3 K_3\right)<0\\
\left(A_4-B_4 K_4\right)^TP+P\left(A_4-B_4 K_4\right)<0\\
P>0
\end{aligned}
\end{eqnarray}
The retrieved $P$ matrix is:
\begin{eqnarray}
\begin{aligned}
P&=\begin{bmatrix}
0.0031 & 0.0001 & 0.0007\\
0.0001 & 0.0031 & -0.0007\\
0.0007 & -0.0007 & 0.4931\\
\end{bmatrix}
\end{aligned}
\end{eqnarray}
The eigenvalues of the $P$ matrix are:
\begin{eqnarray}
\begin{aligned}
\lambda_i&=\begin{bmatrix}
0.0030\\
0.0031\\
0.4931
\end{bmatrix}
\end{aligned}
\end{eqnarray}
Which proves $P$ to be positive definite.
\newpage
The script used to calculate the LMI's is proved here:
\begin{lstlisting}
%% Parameters

c1{1} = 1e-8;
c1{2} = 0.2;
c2{1} = 1e-8;
c2{2} = 0.2;
c3 = 1e-8;

h10 = 0.31;
h20 = 0.31;
h30 = 0.30;

Q10 = 3e-5;
Q20 = 3e-5;

Diameter = 0.14; %in meters
A = ((Diameter^2)*pi)/4; %in meters squared

Q = diag([10 10 10]);
R = diag([1 1]);
horizon = 20;

for i = 1:2
    for j = 1:2
%% Continuous model
    Ac{i}{j} = [-sign(h10-h30)*c1{i}/(2*sqrt(abs(h10-h30))) 0 sign(h10-h30)*c1{i}/(2*sqrt(abs(h10-h30))); 0 -sign(h30-h20)*c2{j}/(2*sqrt(abs(h30-h20)))-c3/(2*sqrt(abs(h20))) sign(h30-h20)*c2{j}/(2*sqrt(abs(h30-h20))); sign(h10-h30)*c1{i}/(2*sqrt(abs(h10-h30))) sign(h30-h20)*c2{j}/(2*sqrt(abs(h30-h20))) -c1{i}/(2*sqrt(abs(h10-h30)))-c2{j}/(2*sqrt(abs(h30-h20)))];
    %Bc{i}{j} = [1/A 0 -sign(h10-h30)*c1{i}*sqrt(abs(h10-h30))+Q10/A; 0 1/A sign(h30-h20)*c2{j}*sqrt(abs(h30-h20))-c3*sqrt(h20)+Q20/A; 0 0 sign(h10-h30)*c1{i}*sqrt(abs(h10-h30))-sign(h30-h20)*c2{j}*sqrt(abs(h30-h20))];
    Bc{i}{j} = [1/A 0;0 1/A;0 0];
    Cc{i}{j} = [1 0 0;0 1 0;0 0 1];
    Dc{i}{j} = 0;
        
    [Klqr{i}{j},S,E] = lqr(Ac{i}{j},Bc{i}{j},Q,R);
    
    Acl{i}{j} = Ac{i}{j}-Bc{i}{j}*Klqr{i}{j};
    end
end

%%%%%%%%%%%%%% LMI's
P=sdpvar(3,3);

L1 = [[Acl{1}{1}'*P+P*Acl{1}{1}]<0];
L2 = [[Acl{1}{2}'*P+P*Acl{1}{2}]<0];
L3 = [[Acl{2}{1}'*P+P*Acl{2}{1}]<0];
L4 = [[Acl{2}{2}'*P+P*Acl{2}{2}]<0];
L5 = [P>0];

L = L1+L2+L3+L4+L5; 

% solve the LMI using SeDuMi:
diagnostics = solvesdp(L);
if diagnostics.problem == 0
 disp('Feasible')
elseif diagnostics.problem == 1
 disp('Infeasible')
else
 disp('Something else happened')
end

Pfin = double(P);
\end{lstlisting}



















































